\documentclass[12pt,a4paper]{article}
\usepackage[utf8]{inputenc}
\usepackage[russian]{babel}
\usepackage{amsmath}
\usepackage{amsfonts}
\usepackage{amssymb}
\usepackage{caption}
\usepackage{subcaption}
\usepackage{graphicx}
\author{Хасан Хафизов}
\title{Децентрализованное управление строем}

\begin{document}
	\maketitle

\section{Постановка задачи}


\section{Механическая модель агента}
Моделью агента является материальная точка с массой $m$. Закон движения:
$$\begin{cases} 
m \ddot{x} = F_x \\
m \ddot{y} = F_y \end{cases}
$$

$\vec{F}$ — сила, действующая на агента, может включать в себя:
$$ \vec{F} = \vec{u} + \vec{W} + \vec{F_{\text{тр}}}$$
Где $\vec{u}$ — управляющее воздействие, $\vec{W}$ — случайные помехи, $\vec{F_{\text{тр}}}$ — сила трения.
\par
В предлагаемом мной алгоритме управления агентов можно разделить на два класса:
\begin{itemize}
	\item интеллектуальный (мастер)
	\item управляемый (миньон)
\end{itemize}
Закон управления для этих двух типов агентов задаётся по-разному.
\subsection{Мастер}
Мастером является агент, для которого желаемый закон движения $S_d$ задаётся оператором извне: это может быть записанная в память агента траектория, целевая позиция или скорость. \par
Фактически, этот агент ничего не знает о существовании других агентов в строю (миньонов). Его задача — выполнение поставленного закона движения, поэтому закон управления:
$$ \vec{u} = \vec{u}(S_d)$$
Рассмотрим конкретный закон управления для движения по некоторой траектории $\vec{tr}(t)$:
$$ \vec{u}_{tr} = \vec{u}_{along} + \vec{u}_{across} $$
Закон управления состоит из двух частей. Первая $\vec{u}_{along}$ отвечает за усилие вдоль траектории, вторая $\vec{u}_{across}$ — поперёк. Направлением для $\vec{u}_{along}$ служит направление вектора между текущим положением агента и следующей точкой траектории.
\begin{center}{\textit{ (Тут будет более подробно о том, как вычисляется следующая точка траектории, о алгоритмах управления на PD регуляторах, которые используются как в $\vec{u}_{along}$, так и в $\vec{u}_{across}$)}}
\end{center}
Пример движения мастера по траектории, задаваемой параметрическим уравнением, где $s$ — параметр: $s \in [0, 5000]$.
$$x(s) = 300 \cdot \cos(\frac{s}{300}); \ y(s) = s$$

\begin{figure}[!htbp]
	\centering
	\begin{subfigure}{.5\textwidth}
		\centering
		\includegraphics[width=1\linewidth]{master-trajectory-0}
		\caption{Исходная траектория и след от движения}
		\label{fig:sub1}
	\end{subfigure}%
	\begin{subfigure}{.5\textwidth}
		\centering
		\includegraphics[width=1\linewidth]{master-trajectory-0-velocity}
		\caption{Скорость движения мастера}
		\label{fig:sub2}
	\end{subfigure}
	\caption{Движение мастера по заданной траектории с заданной скоростью $\upsilon_{desired} = 20 \frac{\text{м}}{\text{с}}$. Расстояния на рисунках задаются в метрах, время в секундах, скорость в $\frac{\text{м}}{\text{с}}$}.
	\label{fig:test}
\end{figure}
\subsubsection{Оценка движения мастера по траектории}
Агент преодолел заданную траекторию за $t = 331 \text{с}$. \\
Проеденное расстояние:
$$ S = \int_{0}^{5000} \sqrt{x'^2(s) + y'^2(s)} \ ds \approx 6051 \text{м} $$ \\
Средняя скорость $\upsilon_{av} = 18.3  \frac{\text{м}}{\text{с}}$. \\
Стационарным режимом движения можно назвать режим, при котором скорость агента колеблется в пределах между $18.8\frac{\text{м}}{\text{с}}$ и $21.5\frac{\text{м}}{\text{с}}$. Более подробно скорость мастера в стационарном режиме можно увидеть на рис. \ref{fig:errors-master}a \\

\begin{figure}[!htbp]
	\centering
	\begin{subfigure}{.5\textwidth}
		\centering
		\includegraphics[width=1\linewidth]{master-trajectory-1-velocity.png}
		\caption{Скорость мастера в стационарном режиме}
		\label{fig:error-master-1}
	\end{subfigure}%
	\begin{subfigure}{.5\textwidth}
		\centering
		\includegraphics[width=1\linewidth]{master-trajectory-error.png}
		\caption{Отклонения мастера от траектории}
		\label{fig:error-master-2}
	\end{subfigure}
	\caption{Иллюстрация отклонений от желаемого закона движения}.
	\label{fig:errors-master}
\end{figure}
Отклонения агента от заданной траектории представлены на рис. \ref{fig:errors-master}b. Выше нуля — отклонения от траектории вправо, ниже нуля — влево. Максимальное отклонение составляет примерно 15 м. \par
Резкие перепады на графике, обозначенные как точки перепада объясняются тем, что в этих точках у траектории изменяется знак первой производной, а агент, всегда остающийся на внутренней части траектории резко оказывается на внешней. Проверим это утверждение. Заданное выше параметрическое уравнение фактически является уравнением $$x = 300 \cdot \cos(\frac{y}{300})$$
Равенство нулю первой производной:
$$\sin(\frac{y}{300}) = 0 \Rightarrow y = 300 \pi n$$
Найдём первые 5 точек траектории, в которой происходит смена знака второй производной: $$M = \Big\{(-300; \ 942), \ (300; \ 1885), \ (-300; \ 2827), \ (300; \ 3770), \ (-300; \ 4712)\Big\}$$
Найдём моменты времени, в которые эти точки будут достигнуты агентом:
$$\hat{L_t} = \Big\{ 61c, \ 118c, \ 177c, \ 234c, \ 292c \Big\}$$
Реальные же моменты времени, в которые просходит резкое изменение величины отклонения от траектории:
$${L_t} = \Big\{ 50c, \ 110c, \ 172c, \ 226c, \ 289c \Big\}$$
То есть, изменение стороны относительно линии траектории по которой движется агент изменяется незадолго до того, как будет изменён знак первой производной траектории. Эта закономерность так же наблюдается на рис. \ref{fig:master-trajectory-changes-2}. \par
\begin{figure}[!htbp]
	\centering
	\includegraphics[width=0.5\linewidth]{master-trajectory-changes-2}
	\caption{Траектория с наложенными на неё точками изменения стороны следования и точками изменения знака первой производной.}
	\label{fig:master-trajectory-changes-2}
\end{figure}

\begin{center}{\textit{ (Тут будет исседование устойчивости алгоритма к случайным возмущениям)}}
\end{center}

Выводы: движение мастера по траектории является точным и предсказуемым. Данный алгоритм управления по траектории с задаваемой скоростью является пригодным для применения.



\subsection{Миньон}
Миньон является ведомым агентом, он не имеет информации о траектории движения. Его задача сводится к следованию к постоянно обновляемой точке траектории, называемой виртуальным лидером.
\begin{center}{\textit{ (Тут будет об алгоритме построения виртуального лидера)}}
\end{center}
\section{Строй}
\begin{figure}[!htbp]
	\centering
	\includegraphics[width=0.7\linewidth]{platoon-trajectory-0}
	\caption{Движение строем мастера и двух миньонов}
	\label{fig:platoon-trajectory-0}
\end{figure}


\end{document}