\documentclass[a4paper, 14pt]{extarticle}
\usepackage[utf8]{inputenc}
\usepackage[russian]{babel}
\usepackage{amsmath}
\usepackage{amsfonts}
\usepackage{amssymb}
\usepackage{caption}
\usepackage{subcaption}
\usepackage{graphicx}
\usepackage{textcomp}
\usepackage{placeins}

\begin{document}
\large{УПРАВЛЕНИЕ ДВИЖЕНИЕМ ГРУППЫ МОБИЛЬНЫХ РОБОТОВ}\par
\bigskip
\textbf{\textit{Аннотация:}} Цель данной работы — изучить существующие решения в области управления движением группы роботов, в частности возможные подходы решения строевой задачи, а так же разработать собственный алгоритм управления строем квадрокоптеров со следующими требованиями: децентрализованность, высокая отказоустойчивость, робастность по отношению к внешним возмущениям и масштабируемость относительно количества роботов в строю. \par
\textbf{\textit{Ключевые слова:}} управление строем, децентрализация, моделирование, квадрокоптер. \par
\vspace{1cm}
\large{MOBILE ROBOTS GROUP CONTROLL}\par
\bigskip
\textbf{\textit{Abstract:}} The purpose of this work is to study existing algorithms of robots group controlling and some solutions of platoon problem. The second purpose is to implement the algorithm that controlls a platoon of quadcopters with the following characteristics: decentralization, 
fault tolerance, robustness with respect to external disturbances and scalability.  \par
\textbf{\textit{Keywords:}} platoon controlling, decentralization, modeling, quadcopter. \par
\end{document}